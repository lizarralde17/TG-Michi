\chapter{Teoría}
\label{chap:teoria}
Escribir resumen sobre los métodos de clasificación y minería de datos

\section{Algoritmos de \textit{Clustering}}

Un \textit{cluster} es un conjunto de objetos que poseen características similares.  El análisis cluster busca particionar un conjunto de objetos en grupos, de tal forma que los objetos de un mismo grupo sean similares y los objetos de grupos diferentes sean disímiles.   

\subsubsection{Medidas De Similaridad}

Reconocer objetos como similares o disimiles es lo más importante para el proceso de clasificación.  Cuando contamos con aspectos cuantitativos, este aspecto de similaridad se encuentra ligado al concepto de métrica o medidas de distancia y cuando contamos con aspectos cualitativos, hablamos de coeficientes de asociación, que se usan para datos en escala nominal.
\ \\
Las medidas de similaridad usadas con más frecuencia son: 
\begin{enumerate}
\item Medidas de distancia
\item Coeficientes de correlación 
\item Coeficientes de asociación
\item Coeficientes de probabilidad
\end{enumerate}

\subsubsection*{Medidas De Distancia}

Las medidas de distancia de uso más frecuente son:

\begin{enumerate}
\item Distancia euclidiana definida por
$$d_{ij} = \sqrt{\sum \limits_{k=1}^{p} (X_{ik}-X_{	jk})^2}$$
\item Distancia Manhattan o distancia por cuadras. Se define por:
$$d_{ij} = \sum \limits_{k=1}^{p} \vert X_{ik}-X_{jk} \vert$$
\item Distancia de Chebychev.  Calcula la discrepancia más grande en alguna de las dimensiones
$$d_{ij} = Max_{k=1...p} \vert X_{ik}-X_{jk} \vert$$
\end{enumerate}

\subsubsection*{Coeficientes De Correlación}

Usamos el coeficiente de correlación para examinar el grado de similitud que hay entre dos variables numéricas.  El coeficiente de correlación $r$, es un valor real entre $-1$ y $1$.  Si $r=1$ las variables están perfectamente correlacionadas, si $r=-1$ la correlación es negativa y si $r=0$ entonces no hay correlación.  Es decir que cuando $r$ es positivo, las variables tienen un comportamiento similar, ambas crecen o decrecen al mismo tiempo.  Y cuando $r$ es negativo si una variable crece la otra decrece. 

El más conocido es el coeficiente de correlación de Pearson, el cual determina el grado de de correlación o asociación lineal entre casos.  Esta definido por:

$$r_{jk}=\dfrac{\sum \limits_{i}(X_{ij}-\overline{X}_{j})(X_{ik}-\overline{X}_{k})}{\sqrt{\sum \limits_{i}(X_{ij}-\overline{X}_{j})^2} \sqrt{\sum \limits_{i}(X_{ik}-\overline{X}_{k})^2}}$$ con $$i= 1,\cdots,p$$
\ \\
Donde $X_{ij}$ es el valor de la variable $i$ para el caso $j$, y $\overline{X}_j$ es la media de todas las variables que definen el caso $j$.  



\subsubsection*{Coeficientes De Asociación}

Se usan principalmente cuando tenemos datos en escala nominal.  Cada variable toma los valores de 0 (ausencia) y 1 (presencia) de un atributo.
 
 \subsubsection*{Coeficientes De Probabilidad}
 
 \subsubsection{Métodos Jerárquicos De Análisis Cluster}

 Los métodos jerárquicos se dividen en aglomerativos y disociativos.  Los métodos aglomerativos comienzan el análisis con tantos grupos como individuos haya.  A partir de estos datos iniciales se van formando grupos, de forma ascendente, hasta que al final del proceso todos los casos tratados quedan en un mismo grupo.\par

 Los métodos disociativos, constituyen el proceso inverso al anterior.  Comienzan con un cluster que contiene todos los casos tratados y a partir de este grupo inicial, a través de varias divisiones, se van formando grupos cada vez más pequeños.  Al final del proceso se tienen tantos grupos como casos. \par 

 A continuación vamos a presentar algunos métodos jerárquicos aglomerativos.
 
 \subsubsection*{Método De La Distancia Mínima}
 
 En este método se considera que la distancia o similitud entre dos clusters viene dada por la mínima distancia (o máxima similitud) entre sus componentes.
 \ \\
 Así, tras efectuar la etapa k-ésima, tenemos ya formados $n-k$ clusters, la distancia entre los clusters $C_{i}$ (con $n_{i}$ elementos) y $C_{j}$ (con $n_{j}$ elementos) sería:
 
$$d(C_{i},C_{j})= \min_{\substack {x_{l}\in C_{i}\\ x_{m}\in C_{j}}}\{d(x_{l},x_{m})\} \,\,\,\,\,\,\,\,\,\, l=1,\cdots,n_{i}; m=1,\cdots,n_{j}$$   
 
En el nivel $k+1$ se unirán los clusters $C_{i}$ y $C_{j}$ si	
$$d(C_{i},C_{j})= \min_{\substack {i_{1},j_{1}=1,...,n-k\\ i_{1}\neq j_{1}}} \{d(C_{i_{1}},C_{j_{1}})\}=$$
$$=\min_{\substack{i_{1},j_{1}=1,\cdots,n-k\\ i_{1}\neq j_{1}}} \left\{\min_{\substack{x_{l}\in C_{i_{1}}\\ x_{m}\in C_{j_{1}}}} \{d(x_{l},x_{m})\}\right\} \,\,\,\,\,\,\, l=1,\cdots,n_{i_{1}}; m=1,\cdots,n_{j_{1}}$$



\subsection{Algoritmo K-Means}

...

\subsubsection{Métodos Jerárquicos}

...


\subsection{Árboles de Decicisón}

...






\clearemptydoublepage
