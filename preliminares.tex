\chapter{Preliminaries}
\label{chap:Preliminaries}
\section{Money Laundering }

Money laundering is a try to give the legality appearance to illegal origin resources and it is a serious problem in our country, mainly because it is related to criminal activities such as drug trafficking, migrant traffic, human trafficking, extortion, illicit enrichment, kidnapping, arms trafficking, corruption and crimes against the financial system.\par
Article 323 of the Criminal Code defines the crime of money laundering as follows:\par
\textit{whoever acquires, safeguards, investments, transports, transforms, safekeeps or manages assets that have remote or immediate origin from activities of migrant smuggling, human trafficking, extortion, illicit enrichment, kidnapping, rebellion, arms trafficking, financing terrorism and activities related to terrorist management, traffic of drugs, narcotics or psychotropic substances, crimes against the financial system, crimes against public administration, or related to the proceeds of crime executed under conspiracy, or give to these resources or goods from such activities the appearance of legality or legalizes them, hide them or disguises their true nature, origin, location, destination, movement or rights to such property or perform any other act to conceal or disguise their illicit origin. (Art.323, Law 599 of 2000)}
\section{Terrorist Financing Sources}
In Colombia it is a crime financing terrorism or managing any resources related to terrorist activities.\par
It is meant by financing of terrorism: \emph{Whoever directly or indirectly provides, collects, delivers, receives, manages, contributes, custodies or saves funds, assets or resources, or perform any other act promoting, organizing, supporting, maintaining, or economically sustaining illegal armed groups or any of their members, or national or foreign terrorist groups, or terrorist activities. (Art. 345, Law 599 of 2000)}.\par
It is meant by the financing of terrorism: \emph{Whoever directly or indirectly provides, collects, delivers, receives, manages, contributes, custodies or saves funds, assets or resources, or perform any other act promoting, organizing, supporting, maintaining, or economically sustaining illegal armed groups or any of their members, or national or foreign terrorist groups, or terrorist activities. (Art. 345, Law 599 of 2000)}.\par

\section{SARLAFT}
It is the management system that must implement the supervised entities by the Colombian Financial Superintendence to protect against the risk of ML/FT and it is implemented through the stages and elements that compose it. (Regulation 022 Colombian Superintendence of Finance).\par
According to Colombian Superintendence of Finance to identify the risk of ML/FT controlled entities as a minimum they must comply:
\begin{enumerate}
  \item To establish methodologies for segmentation of risk factors.
  \item Based on established methodologies, segment the risk factors.
  \item To establish methodologies for risk identification ML/TF and the associated risks for each one of the risk factors.
\item Based on the established methodologies described in the developing of the preceding paragraph, identify ways through, which may arise a risk of ML/TF.
\end{enumerate}
Customers or users represent one of the risk factors that for SARLAFT effects should be taken into account by supervised entities.
According to the Colombian Financial Superintendence, the customer is any natural or legal person with whom the entity establishes and maintains a contractual or legal relationship for supplying any product of its own activity.\par
Taking this into consideration, the purpose of this document is to present a methodology for customer segmentation by using the classification methods theory.

\section{Warning Signs} (From Regulation 022)
Are the set of qualitative and quantitative indicators that ensure timely and/or prospectively identify atypical behaviors of relevant variables, previously determined by the entity.
\section{Segmentation} (From Regulation 022)
Is the process by which takes place the separation of elements in homogeneous groups, within them and heterogeneous between them. The separation is based on the recognition of significant differences in their characteristics (segmentation variables).


