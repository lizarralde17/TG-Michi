\chapter{Preliminares}
\label{chap:preliminares}
\section{Lavado de Activos}

Lavar activos es tratar de dar apariencia de legalidad a recursos de origen ilícito y constituye un serio problema en nuestro país, principalmente por que se encuentra ligado a actividades delictivas como el narcotráfico, tráfico de migrantes, trata de personas, extorsión, enriquecimiento ilícito, secuestro, tráfico de armas, corrupción y delitos contra el sistema financiero.\\[3mm]

El artículo 323 del  código penal define el delito de lavado de activos así:\\

\textit{el que adquiera, resguarde, invierta, transporte, transforme, custodie o administre bienes que tengan su origen mediato o inmediato en actividades de tráfico de migrantes, trata de personas, extorsión, enriquecimiento ilícito, secuestro extorsivo, rebelión, tráfico de armas, financiación del terrorismo y administración de recursos relacionados con actividades terroristas, tráfico de drogas tóxicas, estupefacientes o sustancias sicotrópicas, delitos contra el sistema financiero, delitos contra la administración pública, o vinculados con el producto de delitos ejecutados bajo concierto para delinquir, o les dé a los bienes provenientes de dichas actividades apariencia de legalidad o los legalice, oculte o encubra la verdadera naturaleza, origen, ubicación, destino, movimiento o derecho sobre tales bienes o realice cualquier otro acto para ocultar o encubrir su origen ilícito. (art.323, ley 599 del 2000)}

\section{Financiación del Terrorismo}

En Colombia es un delito financiar el terrorismo o administrar recursos relacionados con actividades terroristas.\\[3mm]

Se entiende por financiación del terrorismo: \emph{El que directa o indirectamente provea, recolecte, entregue, reciba, administre, aporte, custodie o guarde fondos, bienes o recursos, o realice cualquier otro acto que promueva, organice, apoye, mantenga, financie o sostenga económicamente a grupos armados al margen de la ley o a sus integrantes, o a grupos terroristas nacionales o extranjeros, o a terroristas nacionales o extranjeros, o a actividades terroristas. (art. 345, ley 599 del 2000)}.\par 

A diferencia del lavado de activos, en la financiación del terrorismo el orígen de los recursos puede ser lícito.

\section{SARLAFT}

Es el sistema de administración que deben implementar las entidades vigiladas por la Superintendencia Financiera de Colombia para protegerse frente al riesgo de LA/FT y se instrumenta a través de las etapas y elementos que lo integran. (Norma 022 de SuperFinanciera de Colombia)

\ \\
Según la Superintendencia Financiera de Colombia para identificar el riesgo de LA/FT las entidades vigiladas deben como mínimo:
\begin{enumerate}
  \item Establecer las metodologías para la segmentación de los factores de riesgo.
  \item Con base en las metodologías establecidas, segmentar los factores de riesgo.
  \item Establecer las metodologías para la identificación del riesgo de LA/FT y sus riesgos asociados respecto de cada uno de los factores de riesgo.
  \item Con base en las metodologías establecidas en desarrollo del literal anterior, identificar las formas a
      través de las cuales se puede presentar el riesgo de LA/FT.
\end{enumerate}
\ \\
Clientes o Usuarios representan uno de los factores de riesgo que para los efectos del SARLAFT deben tener en cuenta las entidades vigiladas. Según la Superintendencia Financiera de Colombia, el cliente es toda persona natural o jurídica con la cuál la entidad establece y mantiene una relación contractual o legal para el suministro de cualquier producto propio de su actividad.
\ \\
\ \\
Tomando esto en consideración, es el propósito de este trabajo exponer una metodología para la segmentación de clientes haciendo uso de la teoría de métodos de clasificación.


\clearemptydoublepage
