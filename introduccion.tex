\chapter{Introducción}
\label{chap:introduccion}

El objetivo de este trabajo es hacer uso de la teoría de métodos de clasificación para  realizar un modelo aplicado a la gestión de riesgo que permita efectuar la segmentación de clientes vinculados a una entidad financiera.  Esto tomando en cuenta que la Superintendencia Financiera de Colombia considera que los clientes o usuarios son unos de los agentes generadores del riego de lavado de activos y financiación del terrorismo (en adelante LA/FT). La segmentación, según se define en la circular externa 022 del 2007, es el proceso por medio del cual se lleva a cabo la separación de elementos en grupos homogéneos al interior de ellos y heterogéneos entre ellos. La separación se fundamenta en el reconocimiento de diferencias significativas en sus características (variables de segmentación).  Las variables de segmentación que se tendrán en cuenta para los clientes son: actividad económica, volumen o frecuencia de sus transacciones, monto de ingresos y egresos y patrimonio, etc.\par 


Para iniciar veremos las definiciones de lavado de activos y financiación del terrorismo que se encuentran estipuladas en la ley colombiana, que es SARLAFT y que deben hacer las entidades vigiladas para implementarlo.  Después se darán las definiciones pertinentes sobre los principales métodos de clasificación y se intentará mostrar de que manera se puede aplicar la teoría de un método de clasificación seleccionado a la segmentación de clientes que deben efectuar las entidades vigiladas por la superintendencia financiera para identificar el riesgo de LA/FT.\par 


\clearemptydoublepage
