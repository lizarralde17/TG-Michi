\chapter{Introduction}
\label{chap:Introduction}
The purpose of this work is to make use of the classification methods theory to create a model which will be applied to risk management issues and allows effective customer segmentation linked to a financial institution. This taking into account that Colombian Financial Superintendence considers that customers or users are agents of money laundering and terrorist financing sources (hereinafter ML/TF). This segmentation, as defined in the external circular letter 022 of 2007, is the process by which the separation of elements takes place in homogeneous groups, within them and heterogeneous among them.
The separation is based on the recognition of major differences in their characteristics (segmentation variables). The segmentation variables that will be considered for customers are: economic activity, volume or frequency of their transactions, amount of income and assets.
As starting point, we will see the definitions of money laundering and terrorist financing as they are stipulated in Colombian law, what is SARLAFT \footnote{Certification on risk administration system to prevent money laundry and terrorism financing, SARLAFT  for  its abbreviation in Spanish (Sistema de Administración del Riesgo de Lavado de Activos y de la Financiación del Terrorismo)}   and what is requiered by supervised entities to implement this program. After that, will be given the relevant definitions of the main classification methods and there will be a try to show how can the theory of a given classification method be applied to customer segmentation, performed by Financial Superintendence´s supervised entities in order to identify the risk of ML/TF.

